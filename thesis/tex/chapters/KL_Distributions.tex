%! Author = jacobcarlson
%! Date = 4/27/23

\section{Kullback-Leibler Divergence Method Selection}

The Kullback-Leibler Divergence, $D_{KL} (p||q)$, measures the difference of distribution $p$ against distribution $q$
and will be used as the basis of comparison between distributions in this paper.
In the context of differentiating a set of distributions against a base standard, it made sense to consider the
distribution, or set of distributions, from the given file that we want to classify as $p$ and the ground truth
distribution sets as $q$.
This was how all previous accuracies were determined.

\subsection{Symmetric KL Divergence | Exploring Flipped Divergence}
$D_{KL}$ is asymmetric, and flipped $D_{KL}$  can be used to provide different insights than the standard method.
Flipping the KL Divergence, $D_{KL}(q||p)$, as well as taking the average of both options,
$(D_{KL}(p||q)+D_{KL}(q||p)) / 2$, provides alternative values that could provide different
and possible valuable information.
Table \ref{tab:kl1} shows the probability of the new KL Divergence methods being more accurate than $D_{KL}(q||p)$,
tested using the Wilcoxon Signed Rank test.
Both options have a high probability of being more accurate than $D_{KL}(x||dist)$, but neither are conclusively
better since they don’t meet the level of significance laid out in this test $(a=0.05)$.

\begin{table}[H]
    \begin{center}
        \captionsetup{justification=centering}
        \caption{Probability of KL Divergence method yielding higher accuracy than $D_{KL}(x||dist)$}
%        \footnotesize{Z scores and Probabilities of a more accurate model when using a different KL Divergence
%        method compared to $D_{KL}(x||dist)$, using the differences between the accuracies of samples tested on the
%        same data under the same conditions.}
%        \footnotesize{\\$Opcode Set: Malicious Only,$}
        \begin{tabular}{l|S|S}
            \textbf{KL Method} & \textbf{Z} & \textbf{p}\\
            \hline
            $D_{KL}(dist||x)$ & 0.98 & 0.837\\
            Symmetric & 1.06 & 0.856\\
        \end{tabular}
        \label{tab:kl1}
    \end{center}
\end{table}

\subsection{Logarithm Mapping KL Divergence Values}
KL Divergence values range from zero, when two distributions are exactly alike, to infinity, when distributions
share no similarity.
While the KL divergence values derived from these distributions fall in a much smaller range with an upper bound
near $~10$.
Machine learning algorithms typically perform better when input data is normalized to a standard range\cite{singh}.
To normalize and more evenly distribute the data, the KL Divergence value was mapped using $log_{10}$ to put almost
all the data in between $-1$ and $+1$.

On the lower bound, no samples were similar enough to a ground truth distribution to result in a KL Divergence
of less than $.1$, so no samples passed the lower bound of $-1$.
Any KL Divergence value that was greater than $10$ will pass the soft boundary of $+1$ after mapped.
The number of upper bound outliers is minimal and these values do not range very far above $10$.
Any outliers will be very close to $+1$ so none of these outliers are significant.
Most data ends up within a range of $-1$ and $+1$ so outliers were not a very large concern.

Table \ref{tab:mapComparisons} shows the probability of the accuracy being higher using the transformation $log_{10}$ on
KL Divergence against the lack of a transformation.

\begin{table}[H]
    \begin{center}
        \captionsetup{justification=centering}
        \caption{Probability that taking the $log_{10}$ of the KL Divergence result in higher Accuracy}
%        \footnotesize{Z scores and Probabilities of a higher accuracy when taking the $log_{10}$ of the KL Divergence
%            before being used as input for the model, using the differences between the accuracies of samples tested
%        on the same data under the same conditions.}
%        \footnotesize{\\$Opcode\; Set: Malicious\; Only$}
        \begin{tabular}{l|S|S}
            \textbf{KL Method} & \textbf{Z} & \textbf{p}\\
            \hline
            $D_{KL}(x||dist)$ & -0.75 & 0.226\\
            $D_{KL}(dist||x)$ & 0.79 & 0.786\\
            Symmetric & 0.61 & 0.730\\
        \end{tabular}
        \label{tab:mapComparisons}

    \end{center}
\end{table}

Taking the $log_{10}$ of KL Divergence values did not provide a significant enough increase in accuracy when
compared to the same KL Divergence with no transformation.

The flipped and symmetric accuracies had above a $84\%$ probability of being more accurate than $D_{KL}(x||dist)$
and both methods had above a $73\%$ probability of being more accurate when used with a transformation.
While neither parameter was significantly more accurate on its own, they were combined and tested against
$D_{KL}(x||dist)$ in Table \ref{tab:xdistComparison}

\begin{table}[H]
    \begin{center}
        \captionsetup{justification=centering}
        \caption{Probability of KL transformations \\against $D_{KL}(x||dist)$}
%        \footnotesize{Z scores and Probabilities of different KL divergence methods being more accurate than
%            $D_{KL}(x||dist)$, using the differences between the accuracies of samples tested on the same data
%            under the same conditions.}
%        \footnotesize{\\$Opcode\; Set: Malicious\; Only$}
        \begin{tabular}{l|S|S}
            \textbf{KL Method} & \textbf{Z} & \textbf{p}\\
            \hline
            $log_{10}(D_{KL}(dist||x))$ & 2.18 & 0.985\\
            $log_{10}(Symmetric)$ & 0.78 & 0.784\\
        \end{tabular}
        \label{tab:xdistComparison}
    \end{center}
\end{table}

$log_{10}(D_{KL}(dist||x))$ proved to be more accurate than $D_{KL}(x||dist)$ with a high level of significance
when compared with the Wilcoxon signed rank test in Table \ref{tab:xdistComparison}.
$log_{10}(D_{KL}(dist||x))$ will now be the primary KL Divergence method.

This did not change any findings made in the previous sections.
While the accuracies of the opcode tests increased they did so consistently across all opcodes sets
and did not change how they compare.



