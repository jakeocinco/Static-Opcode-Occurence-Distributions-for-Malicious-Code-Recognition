%! Author = jacobcarlson
%! Date = 4/26/23



\section{Proposed Plan}

Static malware analysis is an approach in which an executable is inspected for malware without being executed. The
most common methods include matching common patterns or deriving a series of parameters to learn from each file,
such as the Ember Data set\cite{ember}.
This paper proposes a method of static malware analysis in which an executable will be translated into a set of distributions that will be compared against ground truth distributions from both
benign and malicious code samples.
These comparisons will then be used to determine whether a file is malicious or
benign.

All compiled code exists as a sequence of op codes, instructing the machine on what to do at the most basic level.
While the sequence in which the opcodes are written is not the exact same order in which they will be executed, it
is still a very strong representation of the program at hand.
To create a set of distributions for an executable,
the number of operations between occurrences of the same opcode will be obtained for each common opcode.
This data is then placed into bins and divided by the total number occurrences to create an empirical
probability distribution of the occurrences of each tracked opcode.

This distribution will then be compared to ground truth distributions using the Kullback-Leibler divergence to
assess the difference between distributions.
These KL divergence values will then be used as input data in a machine learning model to detect the
presence of malicious code in an executable.

