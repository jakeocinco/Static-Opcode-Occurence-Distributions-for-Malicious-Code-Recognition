%! Author = jacobcarlson
%! Date = 4/25/23

\section{Data Acquisition}

To train a network capable of detecting malicious code, a large dataset was needed. Most online datasets for malware
detection do not provide raw files in their datasets, especially at scale; however, Practical Security Analytics,
an online security blog, has released a dataset of $200,000+$ benign and malicious executables for machine learning
purposes\cite{lester}.

This analysis focuses on the instructions that a machine will execute, so Unix command $objdump\; -d$ was used to
disassemble the executable into a human readable form that lists the instructions as they would be received by the
operating system executing them. This form provides more than the opcode being executed, so extra information such
as the line number and the parameters of the instruction must be removed. Each instruction follows a common form,
so python was used to filter each entry down to its opcode. To remove the risk of the executable being used and
simplify data access, opcode lists will then be stored and used for future operations instead of the executable form.

